\documentclass{article}
\usepackage{siunitx}
\usepackage{mathtools}
\usepackage{enumerate}
\usepackage{graphicx}
\usepackage{amssymb}
\sisetup{load-configurations = abbreviations}

\usepackage{fullpage}

\begin{document}

\begin{center}
\textsc{\Large ECE 454 Assignment 3}\\[0.5cm]
\textsc{Amir Benham 20393292, Andrew Svoboda 20369388}\\[0.5cm]
\end{center}

\begin{enumerate}

	\item 

	\item 
	By induction. Where the hypothesis is: a connected graph with one node has no edges.
	
	When there are \(k\) nodes, we have there are \(k-1\) edges. Show that with \(k+1\) nodes, there are \(k\) edges. 

	We can say that a connected graph \(G\) has \(k+1\) nodes, and take one of the leaf nodes of the tree. We know that a leaf node can only have one edge. If we remove this leaf node and corresponding edge, \(G\) is still a tree. Thus, \(G\) now has \(k\) nodes, and \(k - 1\) edges.

	In effect, if one removes each node and the associated edge, then at the end there will only exist the root node without any other edges. 

	\item 


	\item 

	\begin{enumerate}
		\item There are two client processes operating, \(P_a\) and \(P_b\). \(P_a\) requests a read of file \(f\) at time 1.  \(P_b\) requests to write to file \(f\) at time 2. The UNIX notion of correctness states that the read will occur and return the last state of the file, before \(P_b\)'s changes take effect. However, if there is a large delay such that the write of \(P_b\) takes effect before \(P_a\), then \(P_a\) will see the file as it was changed, while operating under the assumption that no such change is in effect.

		Alternatively, this question could be interpreted in the following manner. If the same processes above, \(P_a\) and \(P_b\), both attempt a write operation of the same file at the same time, we can construct a situation where UNIX correctness fails. If \(P_a\) begins the write first, and \(P_a\) finishes the write after \(P_b\), then \(P_b\)'s write will override that of \(P_a\). Essentially the OS uses the start time of the write to assert which should be treated as the correct write, rather than the last write of the file. 

		\item By instead using the end time of the successful write, rather than the start time, to determine the correct write the notion of UNIX correctness is met.

		\item
	\end{enumerate}

	\item 

	\item 
\end{enumerate}

\end{document}
