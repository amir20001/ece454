\documentclass{article}
\usepackage{siunitx}
\usepackage{mathtools}
\usepackage{enumerate}

\usepackage{amssymb}
\sisetup{load-configurations = abbreviations}

\usepackage{fullpage}

\begin{document}

\begin{center}
\textsc{\Large ECE 454 Assignment 2}\\[0.5cm]
\textsc{Amir Benham 20393292, Andrew Svoboda 20369388}\\[0.5cm]
\end{center}

\begin{enumerate}

	\item As we can see in the image below we have a network with 5 peers and we want to go from node 0 to node 7.
		[image q1 here]
	
	In this network the finger table of node 0 would look like: \(FT[0]=4, FT[1]=4, FT[2]=0\) which means our next hop would be to node 4. This means that we would have the following situation.
\[
p=0, q=4 , k=7
\]
\[
q-p=0 \; and\;  k-p=3
\]
\[
q-p \ge \frac{k-d}{2}
\]
\[
0\ge \frac{3}{2}
\]
Therefore the equality does not hold.

\item
 \begin{enumerate}
	\item
	\item Only the predecessor to the item that is being inserted needs to be updated, not including the joining node. For a total of 1 node.
\end{enumerate}
\item For this solution we need to introduce a couple of new 

\item For this solution we need to introduce a few new functions. First we will introduce a\( itob()\) function that converts an integer to its binary representation. We will need the reverse of that function \(btoi()\). We will also need a \(concat()\) function that will combine two binary values ie. \(Concat(100,010)\) should result in 100010.  We will also need the reverse of the concat function, we will call it \(split()\) which takes one binary value and splits it down the middle and returns the two tokens ie. \([b1,b2]=split(someBinaryString)\), where \( b1\) and \(b2\) are the most significant and least significant bits respectively.

\((i,j)\rightarrow n\)
\[
n=btoi(concat(itob(i),itob(j))
\]
\(n\rightarrow(i,j)\)
\[
[b1,b2]=split(n)
\]
\[
i=btoi(b1)
\]
\[
j=btoi(b2)
\]
	
\item
There are two possible ways that a route can be created between Q and P either Q chooses P or P chooses Q. The total number of routes are \( \binom{n-1}{c-1}\)

\centerline{\Large$ \therefore $
\(
 P= \frac {2}{\binom{n-1}{c-1}}
\)
}

\item

\item

\end{enumerate}

\end{document}
